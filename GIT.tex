% Comandos de GIT %				% VIDEOTUTORIAL: Curso Git & GitHub, by pildorasinformaticas. https://www.youtube.com/watch?v=ANF1X42_ae4&list=PLU8oAlHdN5BlyaPFiNQcV0xDqy0eR35aU&index=2 %

Primero: Abrir GIT Bash haciendo click derecho desde la carpeta del proyecto. Si no da la opción, abrir GIT Bash directamente desde el buscador y entrar en la carpeta del proyecto

	git --version		% Muestra la version actual de GIT

	pwd					% Muestra el directorio actual

	ls					% Muestra los archivos del directorio actual, similar a dir del símbolo del sistema

	cd Carpeta			% Entra en la carpeta indicada, la cual está en el directorio actual

	cd ..				% Volver a la carpeta anterior del directorio actual

	cd /PathCarpeta		% Vuelve a la carpeta indicada del directorio actual

	clear				% Limpia la ventana, similar a clc de MATLAB



	git init		% Inicia el seguimiento de GIT del proyecto, creando dos áreas: STAGING AREA o ÁREA DE ENSAYO: área temporal que indica que archivos está siguiendo GIT y en qué estado se encuentran
					%																REPOSITORIO LOCAL: donde se almacenan las "copias" de los archivos

	git status -s	% Muestra el contenido de la carpeta que no esté en el Repositorio Local, indicando con: "??" en ROJO los archivos a los que no hace seguimiento
					%																						 "A" en VERDE los archivos a los que sí hace seguimiento (Agregado al Área de Ensayo)
					%																						 "M" en ROJO los archivos que han sido modificados (el contenido ha cambiado respecto a lo guardado en el Repositorio Local)
					%																						 "M" en VERDE los archivos que han sido modificados y están en el Área de Ensayo

	git add Archivo.extensión	% Añade el archivo al Área de Ensayo
	git add .					% Añade al Área de Ensayo todos los archivos de la carpeta que no estén en el Repositorio Local (sin modificar)

	git commit -m "Descripción de la Actualización"		% Añade el archivo al Repositorio Local desde el Área de Ensayo, dándole el nombre deseado, que normalmente será una descripción de la actualización.
	git commit -am "Descripción de la Actualización"	% Hace el add y el commit de una, pero solo cuando hay una modificación. Si no se ha hecho nunca el primer add, no sirve
	
	Si es la primera vez que se hace el commit en el ordenador, pedirá un usuario y un mail:
	git config --global user.username "Fernando"
	git config --global user.email "f.perezs@alumnos.upm.es"

	git log --oneline	% Muestra un listado de todas las copias (commit) y etiquetas (tags) del Repositorio Local, así como sus hashes

	git reset --hard HashDeLaCopia	% Restaura el archivo deseado a la versión indicada por su código (hash), y elimina del Repositorio Local todas las copias posteriores a esa

	git commit --amend		% Abre el editor VIM para modificar la descripción del último commit
		:i		% Para empezar a editar
		Se pulsa la tecla suprimir para borrar todo
		Se pulsa la tecla escape para salir del modo edición
		:i		% Para volver a editar
		Se escribe la descripción deseada
		Se pulsa la tecla enter
		Se pulsa la tecla escape para salir del modo edición
		:wq		% Guarda los cambios y sale del editor
		:qa!	% Sale del editor sin guardar los cambios
	git commit -m "Descripción de la Actualización" --amend		% Lo mismo sin abrir el editor VIM



	git remote add origin LinkDelRepositorioGitHub.git		% Crea la direccion de un repositorio remoto (Repositorio GitHub) en el directorio de repositorios de git
	git branch -M main
	git push -u origin main									% Sube el contenido del Repositorio Local al Repositorio GitHub

	git pull							% Proceso inverso a push. Baja el contenido del Repositorio GitHub al Repositorio Local
	git pull LinkDelRepositorioGitHub	% Alternativa si da error. Por defecto coge la configuración "remote".

	git clone LinkDelRepositorioGitHub.git		% Clona el Repositorio GitHub en la carpeta actual


	git tag NombreIdentificativo -m "Descripción"				% Crea la etiqueta en el Repositorio Local. Las etiquetas son útiles para indicar versiones del proyecto, ya que te aparece el proyecto entero para descargar en un .zip
	git tag NombreIdentificativo HashDelTag -m "Descripción"	% Para copias distintas a la actual

	git tag		% Muestra una lista con las distintas etiquetas creadas

	git tag -d NombreIdentificativo

	git push --tags		% Sube al Repositorio GitHub las etiquetas creadas















	git merge NombredelaRama	% Junta las dos ramas. Se debe hacer desde la rama MAIN