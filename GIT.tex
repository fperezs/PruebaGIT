% Comandos de GIT %

Primero: Abrir GIT Bash haciendo click derecho desde la carpeta del proyecto. Si no da la opción, abrir GIT Bash directamente desde el buscador y entrar en la carpeta del proyecto

	git --version		% Muestra la version actual de GIT

	pwd					% Muestra el directorio actual

	ls					% Muestra los archivos del directorio actual, similar a dir del símbolo del sistema

	cd Carpeta			% Entra en la carpeta indicada, la cual está en el directorio actual

	cd ..				% Volver a la carpeta anterior del directorio actual

	cd /PathCarpeta		% Vuelve a la carpeta indicada del directorio actual

	clear				% Limpia la ventana, similar a clc de MATLAB



	git init		% Inicia el seguimiento de GIT del proyecto, creando dos áreas: STAGING AREA o ÁREA DE ENSAYO: área temporal que indica que archivos está siguiendo GIT y en qué estado se encuentran
					%																REPOSITORIO LOCAL: donde se almacenan las "instantáneas" de los archivos

	git status -s	% Muestra el contenido de la carpeta, indicando con: "??" en rojo los archivos a los que no hace seguimiento
					%													 "A" en verde los archivos a los que sí hace seguimiento (Agregado al Área de Ensayo)

	git add			% Añade el archivo al Área de Ensayo

	git commit		%  Añade el archivo al Repositorio Local desde el Área de Ensayo












