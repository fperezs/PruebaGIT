% Comandos de GIT %

Primero: Abrir GIT Bash haciendo click derecho desde la carpeta del proyecto. Si no da la opción, abrir GIT Bash directamente desde el buscador y entrar en la carpeta del proyecto

	git --version		% Muestra la version actual de GIT

	pwd					% Muestra el directorio actual

	ls					% Muestra los archivos del directorio actual, similar a dir del símbolo del sistema

	cd Carpeta			% Entra en la carpeta indicada, la cual está en el directorio actual

	cd ..				% Volver a la carpeta anterior del directorio actual

	cd /PathCarpeta		% Vuelve a la carpeta indicada del directorio actual

	clear				% Limpia la ventana, similar a clc de MATLAB



	git init		% Inicia el seguimiento de GIT del proyecto, creando dos áreas: STAGING AREA o ÁREA DE ENSAYO: área temporal que indica que archivos está siguiendo GIT y en qué estado se encuentran
					%																REPOSITORIO LOCAL: donde se almacenan las "copias" de los archivos

	git status -s	% Muestra el contenido de la carpeta que no esté en el Repositorio Local, indicando con: "??" en ROJO los archivos a los que no hace seguimiento
					%																						 "A" en VERDE los archivos a los que sí hace seguimiento (Agregado al Área de Ensayo)
					%																						 "M" en ROJO los archivos que han sido modificados (el contenido ha cambiado respecto a lo guardado en el Repositorio Local)
					%																						 "M" en VERDE los archivos que han sido modificados y están en el Área de Ensayo

	git add Archivo.extensión	% Añade el archivo al Área de Ensayo

	git commit -m "Descripción de la Actualización"		%  Añade el archivo al Repositorio Local desde el Área de Ensayo, dándole el nombre deseado, que normalmente será una descripción de la actualización.
	
	Si es la primera vez que se hace el commit en el ordenador, pedirá un usuario y un mail:
	git config --global user.username "Fernando"
	git config --global user.email "f.perezs@alumnos.upm.es"

	git log --oneline	% Muestra un listado de todas las copias (commit) del Repositorio Local

	git reset --hard CódigoDeLaCopia	% Restaura el archivo deseado a la versión indicada por su código, y elimina del Repositorio Local todas las copias posteriores a esa








